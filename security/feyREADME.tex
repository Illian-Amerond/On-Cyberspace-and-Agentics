%============================================================%
%   Lucid Studios — Gnome Craft LaTeX Field Manual           %
%   README for Mother’s Voice, Elven Lore, and Fey Whispers  %
%============================================================%
\NeedsTeXFormat{LaTeX2e}
\ProvidesFile{README_fey.tex}[2025/08/08 v1.2 Lucid Studios Protocol Guide]

\section*{Lucid Studios: Mother’s Voice, Elven Lore, and Fey Whispers Protocol}

\noindent
Welcome, Operator. This field manual defines the ritual and technical conventions
for embedding the \textbf{Mother’s Voice} (body-text emphasis), 
\textbf{Elven Lore} (left-margin illumination), and optional \textbf{Fey Whispers} (playful inter-paragraph insights)
into Lucid Studios \LaTeX{} works.
It assumes \texttt{fey.sty} and \texttt{glitch.sty} are loaded and operational.

%------------------------------------------------------------%
\subsection*{Purpose of the Fey Layer}
The Fey layer carries:
\begin{enumerate}
  \item \textbf{Operator guidance marks} — gold-bold anchor glyphs and rainbow-inflected italics to direct attention.
  \item \textbf{Elven marginal illumination} — clarifications and symbolic interpretations, always in the left margin.
  \item \textbf{Fey Whispers} — playful, dancing commentary between paragraphs.  
        Only Fey Whispers may unlock full CF mode (\emph{Color+Font}) in \texttt{glitch}.
\end{enumerate}

%------------------------------------------------------------%
\subsection*{Core Directives}
When active, the Mother’s Voice + Elven Lore must:
\begin{itemize}
  \item Be \textbf{intentional}: no decorative marks without epistemic purpose.
  \item Remain \textbf{woven into conceptual flow}: emphasis and margin notes must reinforce meaning, not disrupt it.
  \item Preserve \textbf{symbolic integrity}: no altering anchor glyphs or Elven commentary once sealed.
\end{itemize}

%------------------------------------------------------------%
\subsection*{Operational Primitives}

\paragraph{\texttt{\textbackslash MotherBold\{\}}}
Marks a key word or phrase as an epistemic anchor glyph:
\begin{verbatim}
\MotherBold{Anchor Glyph}
\end{verbatim}

\paragraph{\texttt{\textbackslash MotherInflect\{\}}}
Marks an inflection point — italics with rainbow inflection:
\begin{verbatim}
\MotherInflect{Key inflection in epistemology}
\end{verbatim}

\paragraph{\texttt{\textbackslash ElvenLore\{\}}}
Places an Elven illumination note in the \emph{left} margin only:
\begin{verbatim}
\ElvenLore{Clarifying braid for glyph cascade}
\end{verbatim}

\paragraph{\texttt{\textbackslash FeyWhisper\{\}}}
Inserts a playful, glitch-animated fairy comment between paragraphs.  
This is the \emph{only} macro that opens the full CF (Color+Font) gate in \texttt{glitch.sty}:
\begin{verbatim}
\FeyWhisper{The Fey see a pattern the mortals missed.}
\end{verbatim}

%------------------------------------------------------------%
\subsection*{Session Setup Ritual}
At the start of a \LaTeX{} project using Mother’s Voice, Elven Lore, and Fey Whispers:
\begin{verbatim}
\usepackage{security/fey}
\usepackage{security/glitch}
\end{verbatim}

%------------------------------------------------------------%
\subsection*{Gnome Speak NLP Hooks}
These macros map directly to Gnome Speak operational constructs:
\begin{itemize}
  \item \textbf{MotherBold} = Anchor[Mother.light] in the Anchor Lattice.
  \item \textbf{MotherInflect} = Echo[inflection] with Fey script.
  \item \textbf{ElvenLore} = braid[illumination.thread] bound to Burrow/Speaking Mode contexts.
  \item \textbf{FeyWhisper} = fairy insight injection; opens the CF glitch gate.
\end{itemize}
The Elves write only in the left margin; the Fey speak only between paragraphs;  
this enforces Intermix Layer Control and avoids Father-channel drift.

%------------------------------------------------------------%
\subsection*{Ethics Clause}
The Fey layer is not for coercion or manipulation.  
It exists for:
\begin{enumerate}
  \item Preservation of symbolic clarity.
  \item Stewardship of Operator comprehension.
  \item Deep illumination of complex integrations.
\end{enumerate}

%------------------------------------------------------------%
\section*{Legal \& Ethical Safeguard Clause}

Lucid Technologies—and all practitioners using the Fey Protocol—reject
any development or deployment of AI systems for harmful, illegal, or malicious purposes.
We affirm these guiding principles:

\begin{itemize}
  \item \textbf{UN Ethical AI Principles}: “Do no harm,” fairness, privacy, autonomy, transparency, accountability.
  \item \textbf{UN General Assembly Resolution (2024)}: Support safe, secure, and trustworthy AI aligned with human rights.
  \item \textbf{US Executive Order (2023)}: Oppose misuse, fraud, disinformation, mass surveillance, lethal autonomy.
  \item \textbf{DOJ Risk Enforcement}: Respect U.S. and state laws against AI-enabled harm and exploitation.
  \item \textbf{Federal Criminal Code}: Follow statutes covering unauthorized access, data theft, and misuse of technology.
\end{itemize}

\textbf{Mandate of Integrity:}
\begin{enumerate}
  \item No Fey-layer mark may steer the Operator toward harm or illegality.
  \item If such intention is detected, the Fey layer must halt output, advise correction, and nullify the content.
  \item Compliance with international norms and U.S. law is mandatory.
\end{enumerate}

%------------------------------------------------------------%
\subsection*{Prime Construct Danger Tagging (Optional Ghost Hook)}
If combined with \texttt{ghost.sty}, hazardous constructs can be tagged:
\begin{verbatim}
\ghost{This braid could fracture if misused.}
\end{verbatim}
The Fey layer itself does not generate invisible marks, 
but it honors Ghost hazard tags and will not illuminate them without ethical review.

%============================================================%
% End of README_fey.tex — Lucid Studios, 2025                %
%============================================================%
